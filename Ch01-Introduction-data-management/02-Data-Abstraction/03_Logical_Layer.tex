%! Author = moustafa
%! Date = 26/04/2020
%---------------------------------------------------------
\VideoClassification
%%%%%%%%%%%%%%%%%%%%%%%%%%%%%%%%%%%%%%%%%%%%%%%%%%%%%%
\begin{frame}
    \frametitle{Logical level}
    \begin{itemize}[<+->]
        \item \textbf{Logical level (Conceptual)}:
        \begin{itemize}[<+->]
            \item Intermediate level
            \item Describes \textbf{\underline{\blue{what}}} data is stored
            \item Describes what the relationship between the stored data is?
            \item It allows you to change the logical view without altering the external view, API, or programs. These change could be
            \begin{itemize}[<+->]
                \item Add a new table
                \item Change the records merge or delete without affecting the running applications
                \item Change attribute (Add,delete) to the existing table
            \end{itemize}
        \end{itemize}
    \end{itemize}
\end{frame}
%%%%%%%%%%%%%%%%%%%%%%%%%%%%%%%%%%%%%%%%%%%%%%%%%%%%%%
\begin{frame}
    \frametitle{Logical level}
    \begin{example}
        \begin{itemize}[<+->]
            \item Database contains product information.
            \item Logical Layer describes
            \begin{itemize}[<+->]
                \item The product fields and their data types
                \item How this product interact with other entities in the database
                \item The programmers' design this level based on business knowledge and the requirements
            \end{itemize}
        \end{itemize}
    \end{example}

\end{frame}


%---------------------------------------------------------


%%%%%%%%%%%%%%%%%%%%%%%%%%%%%%%%%%%%%%%%%%%%%%%%%%%%%%%%%%%%%%%%%%%%%%%%%%%%
%%% Local Variables:
%%% mode: latex
%%% TeX-master: "../main"
% !TeX root = ../main.tex
%%% TeX-engine: xetex
%%% End:
