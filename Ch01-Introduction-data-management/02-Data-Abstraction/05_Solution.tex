%! Author = moustafa
%! Date = 26/04/2020
%---------------------------------------------------------
\VideoClassification[column=2, colour=blue]
%%%%%%%%%%%%%%%%%%%%%%%%%%%%%%%%%%%%%%%%%%%%%%%%%%%%%%
\begin{frame}[c]
    \frametitle{Data solution thinking (Summary) }
    % !!!!!!!!!!!!!! We need to mention that this slide is just an overview there will be a detailed one later
    \begin{center}
        Let's answer our previous question. How can we solve data challenges?
    \end{center}
\end{frame}

%%%%%%%%%%%%%%%%%%%%%%%%%%%%%%%%%%%%%%%%%%%%%%%%%%%%%%

%%%%%%%%%%%%%%%%%%%%%%%%%%%%%%%%%%%%%%%%%%%%%%%%%%%%%%
\begin{frame}
    \frametitle{Data solution thinking (Summary) }
    \begin{itemize}[<+->]
        \item Let's split the problem based on the data layers.
        \begin{itemize}[<+->]
            \item View layer
            \begin{itemize}[<+->]
                \item When we need to add/remove/create new reports, it is usually a view layer.
                \item We don't need to change the logical or physical layer to support the view layer.
            \end{itemize}
        \end{itemize}
    \end{itemize}
\end{frame}

%%%%%%%%%%%%%%%%%%%%%%%%%%%%%%%%%%%%%%%%%%%%%%%%%%%%%%
\begin{frame}
    \frametitle{Data solution thinking (Summary) }
    \begin{itemize}[<+->]
        \item Let's split the problem based on the data layers.
        \begin{itemize}[<+->]
            \item Logical Layer
            \begin{itemize}[<+->]
                \item When you have missing sources into your logical layer, and you need to add this source and its structure.
                \item There is a performance issue in the existing reports, and you need to change the model. \forexample reduce the join by creating a new join table (\textit{materialized view}).
                \item Update the data type or the existing relation, which could help to fix some data or performance issues.
            \end{itemize}
        \end{itemize}
    \end{itemize}
\end{frame}

%%%%%%%%%%%%%%%%%%%%%%%%%%%%%%%%%%%%%%%%%%%%%%%%%%%%%%%
\begin{frame}
    \frametitle{Data solution thinking (Summary) }
    \begin{itemize}[<+->]
        \item Let's split the problem based on the data layers.
        \begin{itemize}[<+->]
            \item Physical Layer
            \begin{itemize}[<+->]
                \item When our problem is hard or impossible to fix by optimizing the query (view)/ logical layer, it is time for physical change.
                \item If we need to change your storage/compression/structure/access technique.
                \item If we need to change the data orientation structure from row to column or key-value storage, It is time to change the physical layer.
            \end{itemize}
        \end{itemize}
    \end{itemize}
\end{frame}

%%%%%%%%%%%%%%%%%%%%%%%%%%%%%%%%%%%%%%%%%%%%%%%%%%%%%%
\begin{frame}
    \frametitle{Data solution thinking (Summary) }
    \begin{itemize}[<+->]
        \item Let's split the problem based on the data layers.
        \begin{itemize}[<+->]
            \item https://beginnersbook.com/2015/04/levels-of-abstraction-in-dbms/
            \item https://www.guru99.com/dbms-data-independence.html
            \item https://www.geeksforgeeks.org/data-abstraction-and-data-independence/
        \end{itemize}
    \end{itemize}
    \end{frame}


%%%%%%%%%%%%%%%%%%%%%%%%%%%%%%%%%%%%%%%%%%%%%%%%%%%%%%%%%%%%%%%%%%%%%%%%%%%%
%%% Local Variables:
%%% mode: latex
%%% TeX-master: "../main"
% !TeX root = ../main.tex
%%% TeX-engine: xetex
%%% End:
