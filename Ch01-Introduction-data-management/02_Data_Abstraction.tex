%---------------------------------------------------------
\VideoClassification[column=2, colour=blue]
%%%%%%%%%%%%%%%%%%%%%%%%%%%%%%%%%%%%%%%%%%%%%%%%%%%%%%
\subsection{Data Abstraction}
\begin{frame}
	\frametitle{Motivation to Data Layers (Use Case)}	
	\input{./Figures/chapter-01/Fig_Motivation_DA.tex}	

\end{frame}
%%%%%%%%%%%%%%%%%%%%%%%%%%%%%%%%%%%%%%%%%%%%%%%%%%%%%%
\begin{frame}
	\frametitle{Motivation to Data Layers (Solution Thinking)}
	
	\begin{itemize}[<+->]
		\item How can we think about a data solution or challenges in the data products?
		\begin{itemize}[<+->]
			\item Requirements analysis.
			\item Identify the problem (challenges).
			\item Think about how to overcome the challenges.
			\item Ask your self the following questions:
			\begin{itemize}[<+->]
				\item Can we solve the problem using the current data structure by adding new features?
				\item What if we enhance/change the data structure or modeling?
				\item Could it help if we change the backend engine \forexample (DBMS system)?
			\end{itemize}			
		\end{itemize}
		\item To answer these questions you need to understand the \textbf{\underline{data layers}}.
	\end{itemize}
	
\end{frame}
%%%%%%%%%%%%%%%%%%%%%%%%%%%%%%%%%%%%%%%%%%%%%%%%%%%%%%
\begin{frame}
	\frametitle{Data Layers (Abstraction)}
	\begin{itemize}[<+->]
		\item Any data product (database) contains multi-layers.
		\item Each layer responsible for different tasks and operations.
		\item Each layer interacts with (hardware or software or mixed).
		\item Eliminate the complexity of data interactions; not all internal processes are shared or available for the user.
		\item The developer for each layer hides irrelevant internal details from the developer (users). 
		\item The process of \textbf{\underline{\blue{hiding}}} irrelevant details from the developer (user) is called data \textbf{\underline{\blue{abstraction}}}.
	\end{itemize}	
\end{frame}
%%%%%%%%%%%%%%%%%%%%%%%%%%%%%%%%%%%%%%%%%%%%%%%%%%%%%%
\begin{frame}
	\frametitle{Data Layers (Abstraction)}
	\begin{definition}
		\textbf{Data Abstraction and Data Independence}: DBMS comprises complex data-structures. To make the system efficient in terms of retrieval of data and reduce complexity in terms of usability of users, developers use abstraction i.e., hide irrelevant details from the users. This approach simplifies database design.
		%https://www.geeksforgeeks.org/data-abstraction-and-data-independence/
	\end{definition}	
	%Capacity of changing in one level without affecting the other levels. Copied but forget from where!!!
	\begin{itemize}[<+->]
		\item There are 3 levels of data abstraction.
		\begin{itemize}[<+->]
			\item Physical Level
			\item Logical/Conceptual Level.
			\item View Level.
		\end{itemize}
	\end{itemize}	
	%TOP TIER, MIDDLE TIER, BOTTOM TIER
\end{frame}
%%%%%%%%%%%%%%%%%%%%%%%%%%%%%%%%%%%%%%%%%%%%%%%%%%%%%%
\begin{frame}
\frametitle{Data Layers (Abstraction)}
\input{./Figures/chapter-01/Fig_Data_Abstraction.tex}
\end{frame}
%---------------------------------------------------------
\VideoClassification
%%%%%%%%%%%%%%%%%%%%%%%%%%%%%%%%%%%%%%%%%%%%%%%%%%%%%%
\begin{frame}
	\frametitle{Physical level}
	\begin{itemize}[<+->]
		\item \textbf{Physical level (Internal)}: 
		\begin{itemize}[<+->]
			\item Lowest level.
			\item Describes \textbf{\underline{\blue{how}}} data is stored.
			\item Describes the data structure.
			\item It allows you to modify the lowest level (Physical part) without any change in the logical schema. These change could be
				\begin{itemize}[<+->]				
					\item Using a new storage device
					\item Change the structure of the data used for storage
					\item Change the file type or use a different storage structure
					\item Chang the access method
					\item Modify indexes
					\item Change the compression algorithm or hashing technique.
			\end{itemize}									
		\end{itemize}		
		%https://beginnersbook.com/2015/04/levels-of-abstraction-in-dbms/		
		%https://www.guru99.com/dbms-data-independence.html
	\end{itemize}	
\end{frame}
%%%%%%%%%%%%%%%%%%%%%%%%%%%%%%%%%%%%%%%%%%%%%%%%%%%%%%
\begin{frame}
	\frametitle{Physical level}
	\begin{example}		
		\begin{itemize}[<+->]
			\item Database contains product information.
			\item Physical layer describes
			\begin{itemize}[<+->]
				\item Storage mechanism and the blocks (bytes, gigabytes, terabytes, etc.).
				\item The amount of memory used.
				\item Usually this layer abstracted from the programmers.
			\end{itemize}
		\end{itemize}
	\end{example}
	
\end{frame}
%---------------------------------------------------------
%---------------------------------------------------------
\VideoClassification
%%%%%%%%%%%%%%%%%%%%%%%%%%%%%%%%%%%%%%%%%%%%%%%%%%%%%%
\begin{frame}
	\frametitle{Logical level}
	\begin{itemize}[<+->]
		\item \textbf{Logical level (Conceptual)}: 
		\begin{itemize}[<+->]
			\item Intermediate level
			\item Describes \textbf{\underline{\blue{what}}} data is stored
			\item Describes what the relationship between the stored data is?
			\item It allows you to change the logical view without altering the external view, API, or programs. These change could be
			\begin{itemize}[<+->]
				\item Add a new table
				\item Change the records merge or delete without affecting the running applications
				\item Change attribute (Add,delete) to the existing table
			\end{itemize}									
		\end{itemize}		
	\end{itemize}	 
\end{frame}
%%%%%%%%%%%%%%%%%%%%%%%%%%%%%%%%%%%%%%%%%%%%%%%%%%%%%%
\begin{frame}
	\frametitle{Logical level}
	\begin{example}
		\begin{itemize}[<+->]
			\item Database contains product information.
			\item Logical Layer describes
			\begin{itemize}[<+->]
				\item The product fields and their data types
				\item How this product interact with other entities in the database
				\item The programmers' design this level based on business knowledge and the requirements
			\end{itemize}
		\end{itemize}
	\end{example}
	
\end{frame}


%---------------------------------------------------------
%---------------------------------------------------------
\VideoClassification
%%%%%%%%%%%%%%%%%%%%%%%%%%%%%%%%%%%%%%%%%%%%%%%%%%%%%
\begin{frame}
	\frametitle{View level}
	\begin{itemize}
		\item \textbf{View level (External)}: 
		\begin{itemize}
			\item Highest level.
			\item \textbf{\underline{\blue{View}}} of the data stored? 
			\item Designed for a category of users
			\item The final interface for the user
			\item Extended or hidden based on the user's role
			\item Not all the views is extended to all users, and there is authentication based on the category
		\end{itemize}		
	\end{itemize}	
	
\end{frame}
%%%%%%%%%%%%%%%%%%%%%%%%%%%%%%%%%%%%%%%%%%%%%%%%%%%%%
\begin{frame}
	\frametitle{View level}
	\begin{example}
		\begin{itemize}
			\item The database contains product information
			\item It could be designed to show the sales of the product in a specific region
			\item We might hide information about some products based on the teams or users
		\end{itemize}
	\end{example}
	
\end{frame}

%---------------------------------------------------------
%---------------------------------------------------------
\VideoClassification[column=2, colour=blue]
%%%%%%%%%%%%%%%%%%%%%%%%%%%%%%%%%%%%%%%%%%%%%%%%%%%%%%
\begin{frame}[c]
	\frametitle{Data solution thinking (Summary) }
	% !!!!!!!!!!!!!! We need to mention that this slide is just an overview there will be a detailed one later
        \begin{center}
			Let's answer our previous question. How can we solve data challenges?
        \end{center}
    \end{frame}

%%%%%%%%%%%%%%%%%%%%%%%%%%%%%%%%%%%%%%%%%%%%%%%%%%%%%%

%%%%%%%%%%%%%%%%%%%%%%%%%%%%%%%%%%%%%%%%%%%%%%%%%%%%%% 
\begin{frame}
	\frametitle{Data solution thinking (Summary) }
	\begin{itemize}[<+->]
        \item Let's split the problem based on the data layers.
          \begin{itemize}[<+->]
          \item View layer
            \begin{itemize}[<+->]
            \item When we need to add/remove/create new reports, it is usually a view layer.
            \item We don't need to change the logical or physical layer to support the view layer.
          \end{itemize}
        \end{itemize}
       \end{itemize}
 \end{frame}

%%%%%%%%%%%%%%%%%%%%%%%%%%%%%%%%%%%%%%%%%%%%%%%%%%%%%% 
\begin{frame}
\frametitle{Data solution thinking (Summary) }
	\begin{itemize}[<+->]
        \item Let's split the problem based on the data layers.
          \begin{itemize}[<+->]
           \item Logical Layer
           \begin{itemize}[<+->]
             \item When you have missing sources into your logical layer, and you need to add this source and its structure.
             \item There is a performance issue in the existing reports, and you need to change the model. \forexample reduce the join by creating a new join table (\textit{materialized view}).
             \item Update the data type or the existing relation, which could help to fix some data or performance issues.
            \end{itemize}
           \end{itemize}
        \end{itemize}
 \end{frame}

 %%%%%%%%%%%%%%%%%%%%%%%%%%%%%%%%%%%%%%%%%%%%%%%%%%%%%%%
 %%%%%%%%%%%%%%%%%%%%%%%%%%%%%%%%%%%%%%%%%%%%%%%%%%%%%% 
\begin{frame}
  \frametitle{Data solution thinking (Summary) }
  \begin{itemize}[<+->]
  \item Let's split the problem based on the data layers.
    \begin{itemize}[<+->]
    \item Physical Layer
      \begin{itemize}[<+->]
      \item When our problem is hard or impossible to fix by optimizing the query (view)/ logical layer, it is time for physical change.
      \item If we need to change your storage/compression/structure/access technique.
      \item If we need to change the data orientation structure from row to column or key-value storage, It is time to change the physical layer.
      \end{itemize}
    \end{itemize}
  \end{itemize}
 \end{frame}






%%%%%%%%%%%%%%%%%%%%%%%%%%%%%%%%%%%%%%%%%%%%%%%%%%%%%%%%%%%%%%%%%%%%%%%%%%%%
%%% Local Variables:
%%% mode: latex
%%% TeX-master: "../main"
% !TeX root = ../main.tex
%%% TeX-engine: xetex
%%% End: