%%%%%%%%%%%%%%%%%%%%%%%%%%%%%%%%%%%%%%%%%%%%%%%%%%%%%%
\VideoClassification[column=2, colour=blue]
\subsubsection{Extraction Layer}

\begin{frame}
    \frametitle{Extraction Layer}
    \begin{itemize}[<+->]
		\item In some companies, they hire or dedicate a team for this part (extraction or ingestion team), but in other companies, it is part of the data engineering team.
		\item This layer takes the output analysis and decisions from the previous layer (source system analysis) and implement the extraction (quality from the previous team output highly affect this team).
		\item There is a lot of consideration this team needs to take care of or deal with, but we can summarize it in the following:
		\begin{itemize}[<+->]
			\item Data latency analysis as it affects the tool and the methodology (stream or batch).
			\item Data extraction method (push or pull).
			\item Data size and format compared with the available resources for this project.
	    \end{itemize}
        \item \blue{This layer output is a minimal data cleansing (no transformation) into the staging/landing layer}.
    \end{itemize}
\end{frame}





%%%%%%%%%%%%%%%%%%%%%%%%%%%%%%%%%%%%%%%%%%%%%%%%%%%%%%%%%%%%%%%%%%%%%%%%%%%%
%%% Local Variables:
%%% mode: latex
%%% TeX-master: "../main"
% !TeX root = ../main.tex
%%% TeX-engine: xetex
%%% End: