%%%%%%%%%%%%%%%%%%%%%%%%%%%%%%%%%%%%%%%%%%%%%%%%%%%%%%
\VideoClassification[column=1, colour=blue]
%%%%%%%%%%%%%%%%%%%%%%%%%%%%%%%%%%%%%%%%%%%%%%%%%%%%%%
\midTitle{Dimensions Types: Junk Dimension (Garbage Dimension)}
\begin{frame}
    \frametitle{Junk Dimension}
    \begin{itemize}[<+->]
        %junk assume we have multi flags we will collect them as one id
        \item It used to reduce the number of dimensions (low-cardinality columns) in the dimensional model and reduce the number of columns in the fact table. It is a collection of random transnational codes, flags, or text attributes.
        \item It optimizes space as fact tables should not include low-cardinality or text fields. It mainly includes measures, foreign keys, and degenerate dimension keys.
    \end{itemize}
\end{frame}
%%%%%%%%%%%%%%%%%%%%%%%%%%%%%%%%%%%%%%%%%%%%%%%%%%%%%%
\begin{frame}
    \frametitle{Junk Dimension}
    %\resizebox{\columnwidth}{!}{%
\begin{tikzpicture}[every node/.style={font=\ttfamily}, node distance=1.4in,scale=.6, every node/.style={scale=0.6}]
\matrix  [entity=Car, entity anchor=Car-id]  {
	\properties{
		id,
		colour-id (FK),
		body-id (FK)
	}
};


\matrix  [entity=Colour, left=of Car-id, entity anchor=Colour-id]  {
	\properties{
		id,
		colourname
	}
};
\matrix  [entity=Body, right=of Car-id,xshift=-15ex,entity anchor=Body-id]  {
	\properties{
		id,
		bodyname
	}
};
\draw (-10,1) -- (8,1) node[blue,draw=red, ultra thin, minimum size=1cm] [above,pos=0.5] {Design without junk DIM};
\draw [one to one] (Car-id)  to (Colour-id);
\draw [one to one] (Car-id)  to (Body-id);
\end{tikzpicture}
%}

%\resizebox{\columnwidth}{!}{%
	\begin{tikzpicture}[every node/.style={font=\ttfamily}, node distance=1.4in,scale=.6, every node/.style={scale=0.6}]
	
	\matrix  [entity=Car, entity anchor=Car-id]  {
		\properties{
			id,
			car-attirbute-id (FK),
		}
	};	

	\matrix  [entity=Car-Attributes, left=of Car-id, entity anchor=Car-Attributes-id]  {
		\properties{
			id,
			colourname,
			bodyname
		}
	};
	\draw (-10,1) -- (8,1) node[blue,draw=red, ultra thin, minimum size=1cm] [above,pos=0.49] {Design with junk DIM};
	\draw [one to one] (Car-id)  to (Car-Attributes-id);
	\end{tikzpicture}
%}
%%%%%%%%%%%%%%%%%%%%%%%%%%%%%%%%%%%%%%%%%%%%%%%%%%%%%%%%%%%%%%%%%%%%%%%%%%%
%%% Local Variables:
%%% mode: latex
%%% TeX-master: "../../main.tex"
% !TeX root = ../../main.tex
%%% TeX-engine: xetex
%%% End:

\end{frame}
%%%%%%%%%%%%%%%%%%%%%%%%%%%%%%%%%%%%%%%%%%%%%%%%%%%%%%
\begin{frame}
	\frametitle{Junk Dimension}

	\begin{block}{Junk Dimension Table Size}
		\begin{itemize}
			\item We must split the Junk dimension into more dimensions in case the size grows by the time.
			\item It is easy to calculate the expected number of rows as it is the total number of combinations between the low-cardinality attributes; \forexample 3 columns each have 3 values total = 3 * 3 = 9.
		\end{itemize}
	\end{block}
	
\end{frame}




%%%%%%%%%%%%%%%%%%%%%%%%%%%%%%%%%%%%%%%%%%%%%%%%%%%%%%%%%%%%%%%%%%%%%%%%%%%%
%%% Local Variables:
%%% mode: latex
%%% TeX-master: "../../main"
% !TeX root = ../../main.tex
%%% TeX-engine: xetex
%%% End: