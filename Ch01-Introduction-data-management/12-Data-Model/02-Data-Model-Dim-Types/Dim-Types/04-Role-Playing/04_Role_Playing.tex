%%%%%%%%%%%%%%%%%%%%%%%%%%%%%%%%%%%%%%%%%%%%%%%%%%%%%%
\VideoClassification[column=1, colour=blue]
%%%%%%%%%%%%%%%%%%%%%%%%%%%%%%%%%%%%%%%%%%%%%%%%%%%%%%
\midTitle{Dimensions Types: Role-Playing Dimension}
\begin{frame}
    \frametitle{Role-Playing Dimension}
    \begin{description}
        \item [Role-Playing Dimensions (Re-usable Dimension)] A single physical dimension helps to reference multiple times in a fact table as each reference linking to a logically distinct role for the dimension.
    \end{description}
    \centering
    \begin{tikzpicture}[every node/.style={font=\ttfamily}, node distance=1.4in,scale=.6, every node/.style={scale=0.6}]

    \matrix  [entity=OrderDetail, entity anchor=OrderDetail-OrderID]  {
        \properties{
        OrderID,
        OrderDate (FK),
        ShippedDate (FK),
        DeliveryDate (FK),
        ExpiryDate (FK)
        }
    };
    \matrix  [entity=Calendar, right=of OrderDetail-OrderID, entity anchor=Calendar-id] {
        \properties{
        id,
        date,
        day,
        week,
        month,
        qtr,
        year
        }
    };
    %\draw [one to one] (OrderDetail-OrderID)  to (Calendar-id);
    \draw[densely dotted] (2.2,-1.9)  -- (5.9,-1.9);
    \draw[densely dotted] (2.2,-1.2)  -- (5.9,-1.2);
    \draw[densely dotted] (2.2,-.6 )  -- (5.9,-.6);
    \draw[densely dotted] (2.2,-2.5)  -- (5.9,-2.5);
\end{tikzpicture}
%%%%%%%%%%%%%%%%%%%%%%%%%%%%%%%%%%%%%%%%%%%%%%%%%%%%%%%%%%%%%%%%%%%%%%%%%%%
%%% Local Variables:
%%% mode: latex
%%% TeX-master: "../../main.tex"
% !TeX root = ../../main.tex
%%% TeX-engine: xetex
%%% End:

\end{frame}
%%%%%%%%%%%%%%%%%%%%%%%%%%%%%%%%%%%%%%%%%%%%%%%%%%%%%%
\begin{frame}
    \frametitle{Conformed vs Role-Playing Dimension}
    \begin{block}{Conformed vs Role-Playing}
        \begin{itemize}
            \item \textbf{Conformed} is the same dimension used in different facts and has \textit{\underline{the same meaning}} \forexample CustomerID.
            \item \textbf{Role-Playing} is the same dimension which used multiple times within the same fact but \textit{\underline{with different meanings}} \forexample Date.
        \end{itemize}
    \end{block}
\end{frame}




%%%%%%%%%%%%%%%%%%%%%%%%%%%%%%%%%%%%%%%%%%%%%%%%%%%%%%%%%%%%%%%%%%%%%%%%%%%%
%%% Local Variables:
%%% mode: latex
%%% TeX-master: "../../../../../main"
% !TeX root = ../../../../../main.tex
%%% TeX-engine: xetex