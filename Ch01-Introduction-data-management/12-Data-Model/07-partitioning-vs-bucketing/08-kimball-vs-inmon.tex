%%%%%%%%%%%%%%%%%%%%%%%%%%%%%%%%%%%%%%%%%%%%%%%%%%%%%%
%\VideoClassification[column=2, colour=blue]
%%%%%%%%%%%%%%%%%%%%%%%%%%%%%%%%%%%%%%%%%%%%%%%%%%%%%%
\midTitle{Kimball vs Inmon}
%%%%%%%%%%%%%%%%%%%%%%%%%%%%%%%%%%%%%%%%%%%%%%%%%%%%%%
\begin{frame}
	\frametitle{How to architect the data warehouse?}
	\begin{itemize}[<+->]
		\item Kimball paradigm (Dimensional Modeling).
		\item Inmon paradigm (Enterprise Warehouse).
	\end{itemize}
\end{frame}
%%%%%%%%%%%%%%%%%%%%%%%%%%%%%%%%%%%%%%%%%%%%%%%%%%%%%%
\begin{frame}
	\frametitle{How to architect the data warehouse? }
	\begin{itemize}[<+->]
	\item Architecture focus on the way we construct the data model.
	\item We choose the architecture based on the use case and the project.
	\item The type of business and project time frame are considered before choosing the architecture. 
\end{itemize}
	
\end{frame}
%%%%%%%%%%%%%%%%%%%%%%%%%%%%%%%%%%%%%%%%%%%%%%%%%%%%%%
%%%%%%%%%%%%%%%%%%%%%%%%%%%%%%%%%%%%%%%%%%%%%%%%%%%%%%
\begin{frame}
	\frametitle{Commons between Kimball and Inmon}
		\begin{itemize}[<+->]
		\item \textbf{Both} consider DWH as the central data repository for the enterprise.
		\item \textbf{Both} using ETL to load DWH.
		\item \textbf{Both} serve enterprise reporting needs.
		\item \textbf{Both} combine the DWH elements to build their model.
	\end{itemize}

\end{frame}
%%%%%%%%%%%%%%%%%%%%%%%%%%%%%%%%%%%%%%%%%%%%%%%%%%%%%%
\begin{frame}
	\frametitle{Kimball's paradigm}
	\begin{itemize}[<+->]
		\item Kimball focus on business (organization department).
		\item It is suitable for organizations which have changes (best choice for startups or mid-organizations).
		\item It works well in the agile and quick delivery for the business.
		\item It focus to make the user  access the data easily.
	\end{itemize}
	
\end{frame}
%%%%%%%%%%%%%%%%%%%%%%%%%%%%%%%%%%%%%%%%%%%%%%%%%%%%%%
\begin{frame}
	\frametitle{Kimball's paradigm}
	\begin{itemize}[<+->]
		\item  Data warehouse is a set of small data marts for each department. 
		\item These data mart stored in dimensional model (star or snowflake schema).
		\item According to this approach data warehouse is essentially a union of all data marts.
		\item Ralph Kimball, in 1997, stated that “...\textbf{\textit{the data warehouse is nothing more than the union of all the data marts}}“
	\end{itemize}
	
\end{frame}

%%%%%%%%%%%%%%%%%%%%%%%%%%%%%%%%%%%%%%%%%%%%%%%%%%%%%%
\begin{frame}
	\frametitle{Kimball's paradigm}
	Dimentional Model Architecture Approach:
	\begin{itemize}[<+->]
		\item After identifying business process or mertics, It starts to design the data mart which serve this business.
		\item Identify dimensions and facts to achieve business mertics.
		\item It is okay to have duplicates in the design.
		\item It doesn't focus on normalization.
	\end{itemize}
	
\end{frame}
%%%%%%%%%%%%%%%%%%%%%%%%%%%%%%%%%%%%%%%%%%%%%%%%%%%%%%
\begin{frame}
	\frametitle{Disadvantages of Kimball's paradigm}
	\begin{itemize}
		\item Because of Kimball's focus on individual business (data marts), This lead to the following points:
		\begin{itemize}[<+->]
			\item Lose the idea of single source of truth because the entire data warehouse is not fully integrated.
			\item Redundant data added into the model.
			\item The model is not help to take the stratigic decisions or it needs too complicated process for getting the BI reports from mutliple data marts. 
		\end{itemize}
	\end{itemize}
	
\end{frame}
%%%%%%%%%%%%%%%%%%%%%%%%%%%%%%%%%%%%%%%%%%%%%%%%%%%%%%
\begin{frame}
	\frametitle{Inmon's paradigm}
	\begin{itemize}[<+->]
		\item Inmon’s paradigm, is to have "\textbf{one version of the truth}," an enterprise has all the information in one place named data warehouse, and data marts source their information from the data warehouse.
	
		\item Bill Inmon responded in 1998 by saying,  “\textbf{\textit{You can catch all the minnows in the ocean and stack them together and they still do not make a whale}}“. 
	\end{itemize}
	
\end{frame}

%%%%%%%%%%%%%%%%%%%%%%%%%%%%%%%%%%%%%%%%%%%%%%%%%%%%%%
\begin{frame}
	\frametitle{Inmon's paradigm}
	\begin{itemize}[<+->]
		\item Inmon’s paradigm is startigic (visionary), and has enterprise-wide reporting.
		\item It has very low data redundancy. So, it is easy for maintainance (change) and well-integrated.

	\end{itemize}
	
\end{frame}
%%%%%%%%%%%%%%%%%%%%%%%%%%%%%%%%%%%%%%%%%%%%%%%%%%%%%%
\begin{frame}
	\frametitle{Inmon's paradigm}
	\begin{itemize}[<+->]
		\item Inmon's paradigm has a great edge and impact for enterprise especial in known industries.
		\item There are lots of mature models ready for enterprises and they most of cases following Inmon strategy.
		 For example, If we plan to build a banking, telecom, or insurance data model.
		
	\end{itemize}
	
\end{frame}
%%%%%%%%%%%%%%%%%%%%%%%%%%%%%%%%%%%%%%%%%%%%%%%%%%%%%%
\begin{frame}
	\frametitle{Disadvantages of Inmon's paradigm}
	\begin{itemize}[<+->]
		\item Initial build is costly.
		\item It is time-consuming at the begining. 
		\item It requires highly skilled data modeling team.
		\item It requires more effort in ETL to build data marts after creation of data warehouse. 
	\end{itemize}
	
\end{frame}
%%%%%%%%%%%%%%%%%%%%%%%%%%%%%%%%%%%%%%%%%%%%%%%%%%%%%%
\begin{frame}
	\frametitle{Kimball vs Inmon}
\begin{table}
    \resizebox{\textwidth}{!}{
	\begin{tabular}{|c|c|c|}
		\hline
		Paradigm & Kimball & Inmon \\
		\hline
		Data integration & Focus on the individual business & enterprise-wide\\
		\hline
		Data orientation &  Business process & Subject \\
		\hline
		Approach & Bottom-Up Approach & Top-Down Approach \\
    	\hline	
		Building time & Less time &  lot of time \\
    	\hline			
		Cost & Iterative steps of effecient cost &  \makecell{High initial cost but\\ development costs are low} \\
    	\hline	
		Maintenance & difficult &  easy \\
    	\hline			
		Model & De-normalized &  Normalized \\
    	\hline			
		Modeling Complicated & generalist team to implement &  It needs specialized team \\
		\hline			
		Organization Size & Startups or Mid-Level & Enterprise \\
		\hline
	\end{tabular}
	}
\vspace{.1cm}
	\caption{Kimball vs Inmon}\label{eval_table}
\end{table}

\end{frame}

\begin{frame}
	\frametitle{DWH Architecture Overview}
		\begin{tikzpicture}[every label/.append style={font=\tiny},regentonne/.style={cylinder,aspect=.7,draw,shape border rotate=90}]
		%\draw[step=.5cm,gray,very thin] (-2,-1) grid (9.5,6);
		%SS recatangle
		\filldraw[draw=blue,thick,rounded corners,fill=white] (-2,-1) rectangle (-.55,6);

		%Staging recatangle
		\filldraw[draw=BurntOrange,thick,rounded corners,fill=white] (-.4,-1) rectangle (.4,6);

		%ETL recatangle		
		\filldraw[draw=Maroon,thick,rounded corners,fill=white] (.55,0) rectangle (2.5,6);
		
		%EDW
		\filldraw[draw=OliveGreen,thick,rounded corners,fill=white] (2.7,0) rectangle (7.7,6);
		
		%BI LAYER
		\filldraw[draw=mauve,thick,rounded corners,fill=white] (7.9,0) rectangle (9.5,6);

		%Metadata LAYER
		\filldraw[draw=ballblue,thick,rounded corners,fill=white] (.55,-1.2) rectangle (9.5,-.7);

		%User Access LAYER
		\filldraw[draw=black,thick,rounded corners,fill=white] (.55,-.6) rectangle (9.5,-.1);

		
   	    \node[text width=2cm,font=\scriptsize] at (-.9,5.6) {Source\\ Systems};
   	    
		\node[database,label=below:CRM, database middle segment={draw=black,fill=aliceblue}, database bottom segment={draw=black, fill=aliceblue}, database top segment={draw=black,fill=aliceblue} ] (s1) at (-1.5,5) {};
		
		\node[database,label=below:ERP, database middle segment={draw=black,fill=aliceblue}, database bottom segment={draw=black, fill=aliceblue}, database top segment={draw=black,fill=aliceblue}] (s2) at (-1.5,4) {};
		\node[database,label=below:MobApp, database middle segment={draw=black,fill=aliceblue}, database bottom segment={draw=black, fill=aliceblue}, database top segment={draw=black,fill=aliceblue}] (s3) at (-1.5,3) {};		
		\node[database,label=below:Billing, database middle segment={draw=black,fill=aliceblue}, database bottom segment={draw=black, fill=aliceblue}, database top segment={draw=black,fill=aliceblue}] (s4) at (-1.5,2) {};		
		\node[database,label=below:WebApp, database middle segment={draw=black,fill=aliceblue}, database bottom segment={draw=black, fill=aliceblue}, database top segment={draw=black,fill=aliceblue}] (s5) at (-1.5,1) {};						
		\node[database,label=below:Other, database middle segment={draw=black,fill=aliceblue}, database bottom segment={draw=black, fill=aliceblue}, database top segment={draw=black,fill=aliceblue}] (s6) at (-1.5,0) {};				

		
		\draw[line width=0.25mm, blue] (s1) -- ([xshift=.5cm]s1.east) -- ([xshift=.5cm]s2.east) -- (s2)  -- ([xshift=.5cm]s2.east) -- ([xshift=.5cm]s3.east) -- (s3)-- ([xshift=.5cm]s3.east) -- ([xshift=.5cm]s4.east) -- (s4) -- ([xshift=.5cm]s4.east) --  ([xshift=.5cm]s5.east) -- (s5) --  ([xshift=.5cm]s5.east) -- ([xshift=.5cm]s6.east) -- (s6);
	
		%connection arrow
		\draw[->,line width=0.5mm, black] ([xshift=.5cm,yshift=-.5cm]s3.east) -- ([xshift=1cm,yshift=-.5cm]s3.east);

		%staging
   	    \node[text width=2cm,font=\scriptsize] at (.7,5.6) {Stg.};
		
		\node[database,label=below:Staging, database middle segment={draw=black,fill=WildStrawberry}, database bottom segment={draw=black, fill=WildStrawberry}, database top segment={draw=black,fill=WildStrawberry}] (stg) at (0,2.5) {};				

		%connection arrow
		\draw[->,line width=0.5mm, black] (stg.east) -- ([xshift=.6cm]stg.east);

		
		%ETL
   	    \node[text width=2cm,font=\scriptsize] at (2.2,5.6) {ETL};
        \pic[draw,fill=Maroon]          (etlS) at (1.45,4.3)   {gear={0.07}{17}{15}};
        \pic[draw,fill=uiborange]       (etlM) at (1.45,2.5)   {gear={0.07}{17}{15}};
        \pic[draw,fill=mygreen]         (etlB) at (1.45,.8)   {gear={0.07}{17}{15}};
   		\draw[line width=0.25mm, gray] (2.03,4.3) -- (2.4,4.3);
  		\draw[line width=0.25mm, gray] (2.03,2.5) -- (2.4,2.5);
  		\draw[line width=0.25mm, gray] (2.03,.8) -- (2.4,0.8);
  		\draw[line width=0.25mm, gray] (2.4,4.3) -- (2.4,0.8);  		
		%connection arrow	
		%middle connection to edw
  		\draw[->,line width=0.5mm, black] (2.4,2.5) -- (3.55,2.5);
  		\draw[->,line width=0.5mm, black] (2.15,4.5) -- (3.6,4.5);
 		\draw[->,line width=0.5mm, black] (4.45,4.5) -- (8,4.5);
		%EDW
   	    \node[text width=4.5cm,font=\scriptsize] at (5.3,5.6) {Enterprise Data Warehouse (EDW)};

     	\node[database,label=below:DWH,database radius=.5cm,database segment height=.3cm		, database middle segment={draw=black,fill=green-yellow}, database bottom segment={draw=black, fill=green-yellow}, database top segment={draw=black,fill=green-yellow}] (A)  at (4,2.6) {};
     	
     	
     	%DataMart
     	\node[database,label=below:Data Mart,database radius=.4cm,database segment height=.2cm, database middle segment={draw=black,fill=spirodiscoball}, database bottom segment={draw=black, fill=spirodiscoball}, database top segment={draw=black,fill=spirodiscoball}]  (B)  at (6.5,4) {};     	
     	\node[database,label=below:Data Mart,database radius=.4cm,database segment height=.2cm, database middle segment={draw=black,fill=spirodiscoball}, database bottom segment={draw=black, fill=spirodiscoball}, database top segment={draw=black,fill=spirodiscoball}]  (C)  at (6.5,2.5) {};
     	\node[database,label=below:Data Mart,database radius=.4cm,database segment height=.2cm, database middle segment={draw=black,fill=spirodiscoball}, database bottom segment={draw=black, fill=spirodiscoball}, database top segment={draw=black,fill=spirodiscoball}]  (D)  at (6.5,1) {};     	
     	%ODS
     	\node[database,label=below:Operational Data Store,database radius=.4cm,database segment height=.2cm, database middle segment={draw=black,fill=harvardcrimson}, database bottom segment={draw=black, fill=harvardcrimson}, database top segment={draw=black,fill=harvardcrimson}] (E) at (4,4.6) {};
    
       \draw[line width=0.25mm,alizarin,->] (A) -- (B);
       \draw[line width=0.25mm,alizarin,->] (A) -- (C);
       \draw[line width=0.25mm,alizarin,->] (A) -- (D);
       \draw[line width=0.25mm,alizarin,->] (E) -- (A);

       \draw[line width=0.25mm,alizarin] (7.3,.9) -- (7.3,4);
       
       \draw[line width=0.25mm,alizarin] (6.9,4) -- (7.3,4);
       \draw[line width=0.25mm,alizarin] (6.9,2.5) -- (7.3,2.5);
       \draw[line width=0.25mm,alizarin] (6.9,.9) -- (7.3,.9);
       
       \draw[line width=0.5mm,black,->] (7.3,2.5) -- (8,2.5);
       
		%BI LAYER
		\node[text width=1.3cm,font=\scriptsize] at (8.8,5.6) {BI Layer};
		\node[inner sep=0pt] (rep) at (8.7,4.6) {\includegraphics[width=.1\textwidth,height=.1\textheight]{./Figures/chapter-01/report.png}};
		\node[yshift=.4cm,below of=rep] {{\tiny Reporting}};
		\node (io) at (8.7,2.5) [io] {{\tiny Analytics}}; 		
		\node (start) at (8.7,0.5) [startstop] {{\tiny Integrations}}; 

		%MDM
		\node at (4.5,-.4) {Metadata Repository Management};
		
		%Access Layer
		\node at (4.5,-1) {User Access Management \faExpeditedssl};
		
	\end{tikzpicture}

\end{frame}
%%%%%%%%%%%%%%%%%%%%%%%%%%%%%%%%%%%%%%%%%%%%%%%%%%%%%%
\begin{frame}
	\frametitle{References}

\vspace{1.5cm}

\begin{itemize}
    	\item The Data Warehouse Toolkit: The Definitive Guide to Dimensional Modeling, 3rd Edition by Ralph Kimball. 
		\item https://www.1keydata.com/datawarehousing/inmon-kimball.html
		\item https://www.geeksforgeeks.org/difference-between-kimball-and-inmon/
		\item http://www.datamartist.com/data-warehouse-vs-data-mart
		\item https://www.astera.com/type/blog/data-warehouse-concepts/
\end{itemize}


\end{frame}

