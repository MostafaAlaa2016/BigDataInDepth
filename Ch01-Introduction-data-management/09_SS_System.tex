%%%%%%%%%%%%%%%%%%%%%%%%%%%%%%%%%%%%%%%%%%%%%%%%%%%%%%
\subsubsection{Source System Integration Process}
\begin{frame}
    \frametitle{Source System Integration Process}
    \begin{itemize}[<+->]
        \item In some companies, they hire or dedicate a team for this part (business analyst, system analyst, data analyst, or demand team).
        \item Before we start, we need to document all the communications into any format.
		\begin{itemize}
            \item Confluence pages, Word, or Excel sheet.
            \item Make the discussion online and put comments to make the history available always.
			\item We need to clarify all the tasks and what is the expected output, \forexample (analysis means to document data structure, format, column names, etc.).
        \end{itemize}
    \end{itemize}

\end{frame}

%%%%%%%%%%%%%%%%%%%%%%%%%%%%%%%%%%%%%%%%%%%%%%%%%%%%%%
\begin{frame}
    \frametitle{Source System Integration Process}
    \begin{itemize}[<+->]
        \item  Requirements gathering. % or business need (It could be DWH unification).
        \item  Identify the stakeholders (Data owner(s)).
        \item  Data Analysis includes but not only (format, latency, and column definitions).
        \item  Check the source system access and perform connectivity assessment.
        \item  Initiate the technical discussion about the best way to ingest the data.
        \item  Data Ingestion method and format.
        \item  Sign or confirmation for every point between the stakeholders.
        \item  \blue{This layer deliver a data analysis (Source system interface ) document}.
    \end{itemize}

\end{frame}
