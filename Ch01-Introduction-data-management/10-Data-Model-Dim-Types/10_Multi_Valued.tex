%%%%%%%%%%%%%%%%%%%%%%%%%%%%%%%%%%%%%%%%%%%%%%%%%%%%%%
\VideoClassification[column=1, colour=blue]
%%%%%%%%%%%%%%%%%%%%%%%%%%%%%%%%%%%%%%%%%%%%%%%%%%%%%%
\midTitle{Dimensions Types: Multi-valued dimensions (Many-To-Many Dimension)}
\begin{frame}
	\frametitle{Multi-valued dimensions}
	\begin{itemize}[<+->]
		\item When the relationships between the dimension member and the fact are many to many which means the dimension members are lower granularity than the facts. 
		\item Fact table should contains one-to-one relationship with the dimension. So, we introduce the \textbf{\textit{Bridge table}} when we need to related multiple dimensions values with one record.
	\end{itemize}
	
	\begin{example}
		\begin{itemize}[<+->]
			\item Patients can have multiple diagnoses.
			\item Students can have multiple majors.
			\item customers can have multiple account.
			\item Authors can have multiple publications.
		\end{itemize}
	\end{example}		
	
\end{frame}
%%%%%%%%%%%%%%%%%%%%%%%%%%%%%%%%%%%%%%%%%%%%%%%%%%%%%%
\begin{frame}
	\frametitle{Multi-valued dimensions}
	\begin{example}[Sales of Articles]
		\begin{itemize}[<+->]
			\item Assume we need to report the sales of article and we have some articles has more than one author.
			\item Each author has weighting factor for each article.
			\item According to the report we need to check each author and associate with the articles they have authored. How can we model this case?
			\item Assume the article has only one author \textit{Moustafa}, and the second article has two authors \textit{Ahmed \& Amr}.
		\end{itemize}
	\end{example}

\begin{table}
	\resizebox{.97\columnwidth}{!}{%
		
		\begin{tabular}{| l | l | l | l |}
			\hline
			ID & Name & Email & Bio \\
			\hline 
			\hline		
			123 & Moustafa   & abc@gability.com & S-Engineer \\
			234 & Ahmed   & def@gability.com & L-Engineer \\
			345 & Amr   & geh@gability.com & S-Manager \\
			\hline
		\end{tabular}
		
		\quad
		
		\begin{tabular}{| l | l | l | l |}
			\hline
			ID & Title & Journal & Price \\
			\hline
			\hline		
			11 & 50 & IEEE& 110.0\\
			22 & 55 & ACM&130.0\\
			\hline
		\end{tabular}
	}
	\caption{author and articles sample data.}
\end{table}

\end{frame}
%%%%%%%%%%%%%%%%%%%%%%%%%%%%%%%%%%%%%%%%%%%%%%%%%%%%%%
\begin{frame}
\frametitle{Multi-valued dimensions (Implementation-1)}
\begin{tikzpicture}[every node/.style={font=\ttfamily}, node distance=1.4in,scale=.7, every node/.style={scale=0.7}]
  
    \matrix  [entity=Author, entity anchor=Author-ID]  {
	    \properties{
	    ID,
	    Name,
	    Email,
	    Bio
	    }
    };

	\matrix  [entity=ArticleSales, right =of Author-ID,xshift=-7ex, entity anchor=ArticleSales-ID]  {
	\properties{
		ID,
		AuthorID,
		ArticleID,
		SalesDt,
		Quantity,
		UnitPrice,
		TotalPrice
	}
	};

	\matrix  [entity=Article,  right =of ArticleSales-ID,xshift=-7ex, entity anchor=Article-ID]  {
	\properties{
		ID,
		Title,
		Journal,
		Price
	}
};


\draw [many to one] (ArticleSales-ID)  to (Article-ID);
\draw [many to one] (Author-ID)  to (ArticleSales-ID);
\end{tikzpicture}

%%%%%%%%%%%%%%%%%%%%%%%%%%%%%%%%%%%%%%%%%%%%%%%%%%%%%%%%%%%%%%%%%%%%%%%%%%%
%%% Local Variables:
%%% mode: latex
%%% TeX-master: "../../main.tex"
% !TeX root = ../../main.tex
%%% TeX-engine: xetex
%%% End:



\end{frame}
%%%%%%%%%%%%%%%%%%%%%%%%%%%%%%%%%%%%%%%%%%%%%%%%%%%%%%
\begin{frame}
\frametitle{Multi-valued dimensions (Implementation-1)}
\begin{tikzpicture}[every node/.style={font=\ttfamily}, node distance=1.4in,scale=.7, every node/.style={scale=0.7}]
  
    \matrix  [entity=Author, entity anchor=Author-ID]  {
	    \properties{
	    ID,
	    Name,
	    Email,
	    Bio
	    }
    };

	\matrix  [entity=ArticleSales, right =of Author-ID,xshift=-7ex, entity anchor=ArticleSales-ID]  {
	\properties{
		ID,
		AuthorID,
		ArticleID,
		SalesDt,
		Quantity,
		UnitPrice,
		TotalPrice
	}
	};

	\matrix  [entity=Article,  right =of ArticleSales-ID,xshift=-7ex, entity anchor=Article-ID]  {
	\properties{
		ID,
		Title,
		Journal,
		Price
	}
};


\draw [many to one] (ArticleSales-ID)  to (Article-ID);
\draw [many to one] (Author-ID)  to (ArticleSales-ID);
\end{tikzpicture}

%%%%%%%%%%%%%%%%%%%%%%%%%%%%%%%%%%%%%%%%%%%%%%%%%%%%%%%%%%%%%%%%%%%%%%%%%%%
%%% Local Variables:
%%% mode: latex
%%% TeX-master: "../../main.tex"
% !TeX root = ../../main.tex
%%% TeX-engine: xetex
%%% End:



\begin{table}
	\resizebox{.97\columnwidth}{!}{%
		\begin{tabular}{| l | l | l | l | l | l| l |}
			\hline
			ID & AuthorID & ArticleID & SalesDt & Quantity & UnitPrice & TotalOrder \\
			\hline 
			\hline		
			1 & 123  & 11 & 20200303 & 3 & 10 & 30 \\
			2 & 234  & 22 & 20200304 & 1 & 20 & 20 \\
			3 & 345  & 22 & 20200304 & 1 & 20 & 20 \\
			\hline
		\end{tabular}
	}
	\caption{Output of wrong implementation of ArticleSales}
\end{table}

What are the problems in this implementation?
\begin{itemize}
	\item We can't get the weighting factor for each author.
	\item Duplicated numbers in sales due to many authors has the same article id.
	
\end{itemize}

\end{frame}
%%%%%%%%%%%%%%%%%%%%%%%%%%%%%%%%%%%%%%%%%%%%%%%%%%%%%%
\begin{frame}
\frametitle{Multi-valued dimensions (Implementation-2)}
\begin{tikzpicture}[every node/.style={font=\ttfamily}, node distance=1.4in,scale=.7, every node/.style={scale=0.7}]

\node [cloud,draw,inner ysep=.5em,inner xsep=.5em,fill=blue!5,thick] at (-8,-3.8)  {\makecell[l]{Weghting \faCheckSquareO\\
Duplicate \faBug}};

\matrix  [entity=Author, entity anchor=Author-ID]  {
	\properties{
		ID,
		Name,
		Email,
		Bio
	}
};
\matrix  [entity=AuthorGroupRelation,left=of Author-ID,entity anchor=AuthorGroupRelation-ID]  {
	\properties{
		ID,
		AuthorID,
		WieghtingFactor
	}
};

\matrix  [entity=ArticleSales, below  =of Author-ID,xshift=-12ex,yshift=16ex,entity anchor=ArticleSales-ID]  {
	\properties{
		ID,
		AuthorGroupRelationID,
		ArticleID,
		SalesDt,
		Quantity,
		UnitPrice,
		TotalPrice
	}
};

\matrix  [entity=Article,  right =of ArticleSales-ID,xshift=-8ex, entity anchor=Article-ID]  {
	\properties{
		ID,
		Title,
		Journal,
		Price
	}
};



\draw [one to many] (Author-ID)  to (AuthorGroupRelation-ID);
\draw [many to one] (AuthorGroupRelation-ID) to (ArticleSales-ID);
\draw [many to many] (ArticleSales-ID)  to (AuthorGroupRelation-ID);
\draw [many to one] (ArticleSales-ID)  to (Article-ID);
\end{tikzpicture}

%%%%%%%%%%%%%%%%%%%%%%%%%%%%%%%%%%%%%%%%%%%%%%%%%%%%%%%%%%%%%%%%%%%%%%%%%%%
%%% Local Variables:
%%% mode: latex
%%% TeX-master: "../../main.tex"
% !TeX root = ../../main.tex
%%% TeX-engine: xetex
%%% End:

\begin{table}
	\resizebox{.97\columnwidth}{!}{%
		\begin{tabular}{| l | l | l | l | l | l| l |}
			\hline
			ID & AuthorGroupRelationID & ArticleID & SalesDt & Quantity & UnitPrice & TotalOrder \\
			\hline 
			\hline		
			1 & 321  & 11 & 20200303 & 3 & 10 & 30 \\
			2 & 432  & 22 & 20200304 & 1 & 20 & 20 \\
			3 & 543  & 22 & 20200304 & 1 & 20 & 20 \\
			\hline
		\end{tabular}
	}
	\caption{Output of wrong implementation of ArticleSales}
\end{table}

\end{frame}
%%%%%%%%%%%%%%%%%%%%%%%%%%%%%%%%%%%%%%%%%%%%%%%%%%%%%%
\begin{frame}
\frametitle{Multi-valued dimensions (Final Implementation)}
\begin{tikzpicture}[every node/.style={font=\ttfamily}, node distance=1.4in,scale=.7, every node/.style={scale=0.7}]
\node [cloud,draw,inner ysep=.5em,inner xsep=.5em,fill=green!5,thick] at (-12,-3.8)  {\makecell[l]{Weghting \faCheckSquareO\\
		Duplicate \faCheckSquareO}};  
    \matrix  [entity=Author, entity anchor=Author-ID]  {
	    \properties{
	    ID,
	    Name,
	    Email,
	    Bio
	    }
    };
    \matrix  [entity=AuthorGroupRelation, left=of Author-ID,xshift=9ex,entity anchor=AuthorGroupRelation-ID]  {
		\properties{
			ID,
			AuthorKey,
			WieghtingFactor
		}
	};
%
	\matrix  [entity=AuthorGroup, left=of AuthorGroupRelation-ID,xshift=6ex, entity anchor=AuthorGroup-ID]  {
		\properties{
			ID
		}
	};

	\matrix  [entity=ArticleSales, below right =of AuthorGroup-ID,yshift=7ex,xshift=-6ex, entity anchor=ArticleSales-ID]  {
	\properties{
			ID,
			AuthorGroupID,
			ArticleID,
			SalesDt,
			Quantity,
			UnitPrice,
			TotalPrice
	}
	};

	\matrix  [entity=Article,  right =of ArticleSales-ID, entity anchor=Article-ID]  {
	\properties{
		ID,
		Title,
		Journal,
		Price
	}
};


\draw [one to many] (Author-ID)  to (AuthorGroupRelation-ID);
\draw [many to one] (AuthorGroupRelation-ID) to (AuthorGroup-ID);
\draw [many to one] (ArticleSales-ID)  to (AuthorGroup-ID);
\draw [many to one] (ArticleSales-ID)  to (Article-ID);
\end{tikzpicture}

%%%%%%%%%%%%%%%%%%%%%%%%%%%%%%%%%%%%%%%%%%%%%%%%%%%%%%%%%%%%%%%%%%%%%%%%%%%
%%% Local Variables:
%%% mode: latex
%%% TeX-master: "../../main.tex"
% !TeX root = ../../main.tex
%%% TeX-engine: xetex
%%% End:

\begin{table}
	\resizebox{.97\columnwidth}{!}{%
		\begin{tabular}{| l | l | l | l | l | l| l |}
			\hline
			ID & AuthorGroupID & ArticleID & SalesDt & Quantity & UnitPrice & TotalOrder \\
			\hline 
			\hline		
			1 & 321  & 11 & 20200303 & 3 & 10 & 30 \\
			2 & 432543  & 22 & 20200304 & 1 & 20 & 20 \\
			\hline
		\end{tabular}
	}
	\caption{Expected output of ArticleSales}
\end{table}

\end{frame}
%%%%%%%%%%%%%%%%%%%%%%%%%%%%%%%%%%%%%%%%%%%%%%%%%%%%%%
%multi-valued attributes is to create a relationship between the dimension table and a secondary dimensional table (outrigger table).
\begin{frame}
\frametitle{Example Reference}
	\begin{itemize}[<+->]
		\item Example in this video taken from this link \href{https://www.nuwavesolutions.com/bridge-tables/}{https://www.nuwavesolutions.com/bridge-tables/}
	\end{itemize}
\end{frame}

%%%%%%%%%%%%%%%%%%%%%%%%%%%%%%%%%%%%%%%%%%%%%%%%%%%%%%%%%%%%%%%%%%%%%%%%%%%%
%%% Local Variables:
%%% mode: latex
%%% TeX-master: "../../main"
% !TeX root = ../../main.tex
%%% TeX-engine: xetex
%%% End: