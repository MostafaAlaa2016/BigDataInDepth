%%%%%%%%%%%%%%%%%%%%%%%%%%%%%%%%%%%%%%%%%%%%%%%%%%%%%%
\VideoClassification[column=1, colour=blue]
%%%%%%%%%%%%%%%%%%%%%%%%%%%%%%%%%%%%%%%%%%%%%%%%%%%%%%
\midTitle{Dimensions Types: Multi-valued dimensions (Many-To-Many Dimension)}
\begin{frame}
	\frametitle{Multi-valued dimensions}
	\begin{itemize}[<+->]
		\item When the relationships between the dimension member and the fact are many to many which means the dimension members are lower granularity than the facts. 
		\item Fact table should contains one-to-one relationship with the dimension. So, we introduce the \textbf{\textit{Bridge table}} when we need to related multiple dimensions values with one record.
	\end{itemize}
	
	\begin{example}
		\begin{itemize}[<+->]
			\item Patients can have multiple diagnoses.
			\item Students can have multiple majors.
			\item customers can have multiple account.
			\item Authors can have multiple publications.
		\end{itemize}
	\end{example}		
	
\end{frame}
%%%%%%%%%%%%%%%%%%%%%%%%%%%%%%%%%%%%%%%%%%%%%%%%%%%%%%
\begin{frame}
	\frametitle{Multi-valued dimensions}
	\begin{example}[Sales of Articles]
		\begin{itemize}[<+->]
			\item Assume we need to report the sales of article and we have some articles has more than one author.
			\item According to the report we need to check each author and associate with the articles they have authored. How can we model this case?
		\end{itemize}
	\end{example}
	\begin{tikzpicture}[every node/.style={font=\ttfamily}, node distance=1.4in,scale=.7, every node/.style={scale=0.7}]
\node [cloud,draw,inner ysep=.5em,inner xsep=.5em,fill=green!5,thick] at (-12,-3.8)  {\makecell[l]{Weghting \faCheckSquareO\\
		Duplicate \faCheckSquareO}};  
    \matrix  [entity=Author, entity anchor=Author-ID]  {
	    \properties{
	    ID,
	    Name,
	    Email,
	    Bio
	    }
    };
    \matrix  [entity=AuthorGroupRelation, left=of Author-ID,xshift=9ex,entity anchor=AuthorGroupRelation-ID]  {
		\properties{
			ID,
			AuthorKey,
			WieghtingFactor
		}
	};
%
	\matrix  [entity=AuthorGroup, left=of AuthorGroupRelation-ID,xshift=6ex, entity anchor=AuthorGroup-ID]  {
		\properties{
			ID
		}
	};

	\matrix  [entity=ArticleSales, below right =of AuthorGroup-ID,yshift=7ex,xshift=-6ex, entity anchor=ArticleSales-ID]  {
	\properties{
			ID,
			AuthorGroupID,
			ArticleID,
			SalesDt,
			Quantity,
			UnitPrice,
			TotalPrice
	}
	};

	\matrix  [entity=Article,  right =of ArticleSales-ID, entity anchor=Article-ID]  {
	\properties{
		ID,
		Title,
		Journal,
		Price
	}
};


\draw [one to many] (Author-ID)  to (AuthorGroupRelation-ID);
\draw [many to one] (AuthorGroupRelation-ID) to (AuthorGroup-ID);
\draw [many to one] (ArticleSales-ID)  to (AuthorGroup-ID);
\draw [many to one] (ArticleSales-ID)  to (Article-ID);
\end{tikzpicture}

%%%%%%%%%%%%%%%%%%%%%%%%%%%%%%%%%%%%%%%%%%%%%%%%%%%%%%%%%%%%%%%%%%%%%%%%%%%
%%% Local Variables:
%%% mode: latex
%%% TeX-master: "../../main.tex"
% !TeX root = ../../main.tex
%%% TeX-engine: xetex
%%% End:

\end{frame}

%multi-valued attributes is to create a relationship between the dimension table and a secondary dimensional table (outrigger table).
%



%%%%%%%%%%%%%%%%%%%%%%%%%%%%%%%%%%%%%%%%%%%%%%%%%%%%%%%%%%%%%%%%%%%%%%%%%%%%
%%% Local Variables:
%%% mode: latex
%%% TeX-master: "../../main"
% !TeX root = ../../main.tex
%%% TeX-engine: xetex
%%% End: